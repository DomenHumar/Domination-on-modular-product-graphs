\documentclass[a4paper,12pt]{article}
\usepackage[slovene]{babel}
\usepackage[utf8]{inputenc}
\usepackage[T1]{fontenc}
\usepackage{lmodern}
\usepackage{amsmath}
\usepackage{amssymb}
\usepackage[shortlabels]{enumitem}

\DeclareUnicodeCharacter{03B3}{$\gamma$}
\DeclareUnicodeCharacter{25C7}{$\Diamond$}

\newtheorem{definition}{Definicija}
\newtheorem{primer}{Primer}

\pagestyle{plain}
\begin{document}
\author{Domen Humar in Maja Komic}
\date{November 2023}
\title{
    Domination on modular product graphs \\
    \large Poročilo
    }
\maketitle

\section{Definicije}
\begin{definition}
    Podmnožica vozlišč grafa $G$ (označimo jo z $S$) je \textbf{dominanta podmnožica}
    grafa $G$, če zanjo velja, da je vsako vozlišče grafa $G$, ali znotraj podmnožice $S$,
    ali je sosednjo nekemu vozlišču znotraj $S$.
\end{definition}

\begin{definition}
    \textbf{Dominacijsko število} grafa $G$, označeno z $\gamma(G)$, je velikost/moč/kardinalno število
    najmanjše \emph{dominantne podmnožice} grafa $G$.
\end{definition}

\begin{definition}
    \textbf {Modularni produkt grafov $G$ in $H$} je graf $G\diamond H$ z množico vozlišč $V( G\diamond H) = V(G) \times  V(H)$, ki je 
    unija kartezičnega produkta, direktnega produkta in direktnega produkta komplementov G in H
    $$G \diamond H = G\square H \cup G \times H \cup \overline{G} \times \overline{H} $$.
    Natančneje, točki $(g, h)$ in $(g', h')$ iz grafa $G \diamond H$ sta sosednji, če velja: 
    \begin{enumerate}
        \item če je $g = g'$ in $hh' \in E(H)$; ali 
        \item če je $h = h'$ in $gg' \in E(G)$; ali 
        \item če je $gg' \in E(G)$ in $hh' \in E(H)$; ali
        \item če za $g \neq g'$ in $h \neq h'$ velja $uu’ \notin  E(G)$ in $hh’ \notin  E(H)$. 
    \end{enumerate}
    Povezave iz prve in druge točke so iz kartezičnega produkta, povezave iz tretje točke so iz direktnega produkta
    in povezave iz četrte točke so iz direktnega produkta komplementov.
\end{definition}

\section{Problem}
Naj bosta $G$ in $H$ grafa.
Na različnih primerih grafov želimo preveriti 
spodnjo neenakost in poiskati čim več takih grafov $G$ in $H$ za katera velja enakost.

\begin{equation}
    \gamma(G\diamond H) \leq \gamma (G) + \gamma (H) - 1
\end{equation}

\begin{primer}
    Naj bo $G$ graf z 4 vozlišči in 4 povezavami in $H$ naj bo graf s 3 vozlišči in 
    2 povezavama. Dominacijsko število grafa $G$ je 2, dominacijsko število grafa $H$ pa 1. Ko naredimo modularni produkt na grafih G in H, 
    dobimo graf $G \diamond H$, ki ima dominacijsko število 2. 

    \begin{align*} 
        \gamma(G\diamond H) &\leq \gamma (G) + \gamma (H) - 1 \\ 
        2 &\leq 2 + 1 - 1
    \end{align*}
    
    \noindent Vidimo, da velja neenakost in enakost. Za vizualno reprezentacijo tega primer,
    odpremo in zaženemo Algoritem.sagews v CoCalc-u.

\end{primer}

\section{Reševanje problema}
Za reševanje opisanega problema sva uporabila programski jezik \emph{Python} in okolje Sage (SageMath) znotraj spletne platforme CoCalc.
Najprej sva s pomočjo zgornjih definicij definirala funkciji $dominacijsko\_stevilo(G, H)$ in $dominacijsko\_stevilo(G)$.
Ter funciji, ki preverjata neenakost (1) oz enakost.

\begin{verbatim}
    # MODULARNI PRODUKT
    def modularni_produkt(G, H):
        A = H.cartesian_product(G)
        B = H.tensor_product(G)
        C = H.complement().tensor_product(G.complement())
        X = A.union(B)
        Y = X.union(C)
        return Y
    
    # DOMINACIJSKO ŠTEVILO
    def dominacijsko_stevilo (G):
        p = MixedIntegerLinearProgram(maximization = False)
        b = p.new_variable(binary = True)
        p.set_objective( sum([b[v] for v in G]) )
        for u in G:
            p.add_constraint( 
                b[u] + sum([b[v] for v in G.neighbors(u)]) >= 1 
                )
        a = p.solve()
         return a

    # FUNKCIJA, KI PREVERI NEENAKOST
    def preveri_neenakost(x, y, z):
        return x <= y + z - 1

    # FUNKCIJA, KI PREVERI ENAKOST 
    def preveri_enakost(x, y, z):
        return x == y + z - 1
\end{verbatim}

Nato sva z uporabo zgornjih funcij izvedla simulacijo, ki na parih grafov z raznimi kombinacijami števila 
vozlišč in povezav, poišče modularni produkt grafa, izračuna dominacijska števila osnovnih grafov in njunega 
modularnega produkta, ter preveri neenakost in enakost. Funkcija $naredi\_matriko(n, m)$ sprejme parametra n in m, 
ki označujeta največje število vozlišč grafov G in H vrne pa matriko, ki ima v vrstici podatke 
število vozlišč grafa G, število povezav grafa G, dominacijsko število grafa G, 
število vozlišč grafa H, število povezav grafa H, dominacijsko število grafa H, 
dominacijsko število modularnega produkta grafov G in H, True, če neenakost (1) velja oziroma False, če neenakost (1) ne velja,
ter True če velja enakost oziroma False, če enakost ne velja. 


\begingroup
\makeatletter
\@totalleftmargin=-5cm
    \begin{small}
        \begin{verbatim}
            # SIMULACIJA
            def naredi_matriko(n, m):
            matrika = [["V(H)", "E(H)", "γ(H)", "V(G)", "E(G)", "γ(G)", "γ(G ◇ H)", "neenakost", "enakost"]]
                for k in range(2, n):
                    for H in graphs(k):
                        for d in range(2, m):
                            for G in graphs(d):
                                Y = modularni_produkt(G, H)
                                y = dominacijsko_stevilo (Y)
                                h = dominacijsko_stevilo (H)
                                g = dominacijsko_stevilo (G)
                                i = preveri_neenakost(y, h, g)
                                j = preveri_enakost(y, h, g)
                                matrika.append([H.order(), H.size(), h, 
                                G.order(), G.size(), g, y, i, j])
                return matrika
        \end{verbatim}
    \end{small}
\endgroup

Za boljšo preglednost definiramo še modificirano simulacijo, ki nam poda tabelo podatkov
samo o grafih za katere nastopi enakost pri neenakosti (1).

\begingroup
\makeatletter
\@totalleftmargin=-5cm
    \begin{small}
        \begin{verbatim}
            # MODIFICIRANA SIMULACIJA
            def naredi_matriko_enakosti(n, m):
            matrika_enakosti = [["V(H)", "E(H)", "γ(H)", "V(G)", "E(G)", "γ(G)", "γ(G ◇ H)"]]
            for k in range(2, n):
                for H in graphs(k):
                    for d in range(2, m):
                        for G in graphs(d):
                            Y = modularni_produkt(G, H)
                            y = dominacijsko_stevilo (Y)
                            h = dominacijsko_stevilo (H)
                            g = dominacijsko_stevilo (G)
                            j = preveri_enakost(y, h, g)
                            if j == True:
                                matrika_enakosti.append([H.order(), H.size(), h, G.order(), G.size(), g, y])
            return matrika_enakosti
        \end{verbatim}
    \end{small}
\endgroup

\section{Opazke}
Z izvedbo zgornjega algoritma na grafih, ki imajo do vključno 5 vozlišč, sva dobila tabelo, ki je v Prilogi 1, iz katere je razvidno, da neenakost (1) velja za vse grafe.
S pomočjo modificirane tabele iz Priloge 2 pa sva opazila, da je zadosten pogoj, da nastopi enakost v neenakosti (1), če ima eden izmed grafov, na katerih delamo modularni produkt,
dominacijsko število 1. \\

Na žalost pa tudi opazimo, da ta pogoj ni nujen, saj obstajajo kombinacije grafov za katere velja, da nobeden izmedn njiju nima dominacijskega števila enakega 1,
a vendar vseeno nastopa enakost (vendar so takšni primeri redkejši). 3 takšne primere si lahko ogledate, ko zaženete Algoritem.sagews znotraj CoCalc-a.
\end{document}
