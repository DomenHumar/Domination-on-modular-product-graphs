\documentclass[a4paper,12pt]{article}
\usepackage[slovene]{babel}
\usepackage[utf8]{inputenc}
\usepackage[T1]{fontenc}
\usepackage{lmodern}
\usepackage{amsmath}
\usepackage{amssymb}
\usepackage[shortlabels]{enumitem}
\usepackage{graphicx}

\newtheorem{definition}{Definicija}

\pagestyle{plain}

\begin{document}
\author{Domen Humar in Maja Komic}
\date{November 2023}
\title{Domination on modular product graphs}
\maketitle

\section{Definicije}

    \begin{definition}
        Podmnožica vozlišč grafa $G$ (označimo jo z $S$) je \textbf{dominanta podmnožica}
        grafa $G$, če zanjo velja, da je vsako vozlišče grafa $G$, ali znotraj podmnožice $S$,
        ali je sosednjo nekemu vozlišču znotraj $S$.
    \end{definition}

    \begin{definition}
        \textbf{Dominacijsko število} grafa $G$, označeno z $\gamma(G)$, je velikost/moč/kardinalno število
        najmanjše \emph{dominantne podmnožice} grafa $G$.
    \end{definition}

    \begin{definition}
        \textbf {Modularni produkt grafov $G$ in $H$} je graf $G\diamond H$ z množico vozlišč $V( G\diamond H) = V(G) \times  V(H)$, ki je 
        unija kartezičnega produkta, direktnega produkta in direktnega produkta komplementov G in H
        $$G \diamond H = G\square H \cup G \times H \cup \overline{G} \times \overline{H} $$.
        Natančneje, točki $(g, h)$ in $(g', h')$ iz grafa $G \diamond H$ sta sosednji, če velja: 
        \begin{enumerate}
            \item če je $g = g'$ in $hh' \in E(H)$; ali 
            \item če je $h = h'$ in $gg' \in E(G)$; ali 
            \item če je $gg' \in E(G)$ in $hh' \in E(H)$; ali
            \item če za $g \neq g'$ in $h \neq h'$ velja $uu’ \notin  E(G)$ in $hh’ \notin  E(H)$. 
        \end{enumerate}
        Povezave iz prve in druge točke so iz kartezičnega produkta, povezave iz tretje točke so iz direktnega produkta
        in povezave iz četrte točke so iz direktnega produkta komplementov.
    \end{definition}



\section{Problem}
Naj bosta $G$ in $H$ grafa.
Na različnih primerih grafov želimo preveriti 
spodnjo neenakost in poiskati čim več takih grafov $G$ in $H$ za katera velja enakost.
\begin{equation}
    \gamma(G\diamond H) \leq \gamma (G) + \gamma (H) - 1
\end{equation}
    
\section{Načrt dela}
Najprej bova implementirala sledeči funciji:
    \begin{itemize}
        \item funkcijo, ki sprejme grafa $G$ in $H$ in vrne modularni produkt $G \diamond H$
        \item funkcijo, ki sprejme graf $K$ in vrne dominantno število (Za manjše grafe bova napisala nek manjši program v Python-u,
        za večje grafe pa bova uporabila neko metahevristkiko).
    \end{itemize}
Pri reševanju problema bova uporabljala Sage paket za Python.
\end{document}