\documentclass[a4paper,12pt]{article}
\usepackage[slovene]{babel}
\usepackage[utf8]{inputenc}
\usepackage[T1]{fontenc}
\usepackage{lmodern}
\usepackage{amsmath}
\usepackage{amssymb}
\usepackage[shortlabels]{enumitem}
\usepackage{graphicx}

\newtheorem{definition}{Definicija}

\pagestyle{plain}

\begin{document}
\author{Domen Humar in Maja Komic}
\date{Noveber 2023}
\title{Domination on modular product graphs  A}
\maketitle

\section{Problem}
Naj bosta $G$ in $H$ grafa. Na različnih primerih grafov želimo preveriti 
spodnjo neenakost in poiskati čim več takih grafov $G$ in $H$ za katera velja ta neenakost 
\begin{equation}
    \gamma(G\diamond H) \leq \gamma (G) + \gamma (H) - 1
\end{equation}

\section{Definicije}

    \begin{definition}
        \textbf {Modularni produkt grafov $G$ in $H$} je graf $G\diamond H$ z množico vozlišč $V( G\diamond H) = V(G) \times  V(H)$, ki je 
        unija kartezičega produkta, neposrednega produkta in neposrednega produkta komplementov G in H
        $$G \diamond H = G\square H \cup G \times H \cup \overline{G} \times \overline{H} $$.
        Natančneje, točki $(g, h)$ in $(g', h')$ iz grafa $G \diamond H$ sta sosednji, če velja: 
        \begin{enumerate}
            \item če je $g = g'$ in $hh' \in E(H)$; ali 
            \item če je $h = h'$ in $gg' \in E(G)$; ali 
            \item če je $gg' \in E(G)$ in $hh' \in E(H)$; ali
            \item če za $g \neq g'$ in $h \neq h'$ velja $(u, u’) \notin  E(G)$ in $(v, v’) \notin  E(H)$. 
        \end{enumerate}
    \end{definition}

    \begin{definition}
        Množica $S\subseteq V(G)$ je \textbf {dominirana množica grafa $G = (V,E)$}, če za vsak $u \in V \backslash S$ obstaja $v \in S$, da je $uv i\in E(G)$. 
    \end{definition}

    \begin{definition}
        \textbf {Dominirano število grafa $G = (V, E)$} je moč najmanjše dominirane množice grafa $G$, označimo ga z $\gamma (G)$.
    \end{definition}

\section{Načrt dela}
Najprej bova implementirala sledeči funciji:
    \begin{itemize}
        \item funcijo, ki sprejme grafa $G$ in $H$ (podana z matriko sosednosti) in vrne podularni produkt $G \diamond H$, ter
        \item funcijo, ki sprejme graf $G \diamond H$ (podan z matriko sosednosti) in vrne najmanjšo dominirano množico grafa in vrne moč te množice.
    \end{itemize}

Nato bova s simulacijo opazovala za katere grafe neenakost (1) velja, ko grafoma $G$ in $H$ postopoma dodajamo ogljišča in povezava. Začela bova 
s preprostima grafoma z dvema ogljiščema in eno povezavo, ter jima sistematično dodajala ogljišča.

Pri reševanju problema bova uporabljala Python.
\end{document}